\documentclass{article}

% Language setting
% Replace `english' with e.g. `spanish' to change the document language
\usepackage[english]{babel}

% Set page size and margins
% Replace `letterpaper' with `a4paper' for UK/EU standard size
\usepackage[letterpaper,top=2cm,bottom=2cm,left=3cm,right=3cm,marginparwidth=1.75cm]{geometry}

% Useful packages
\usepackage{amsmath}
\usepackage{graphicx}
\usepackage[colorlinks=true, allcolors=blue]{hyperref}

\title{SIF3004 Final Year Project Proposal}
\author{Lim Ming Kang U2004991/1}

\begin{document}
\maketitle
\section{Problem Statement}

we want to see what happen at cosmic noon (at z=2-3, the star forming rate is super high).
"dusty" galaxies are very hard to observe, observation is needed to see what happen
based on jcmt, The growth and evolution of the aforementioned galaxy overdensities with redshift is to this day still not characterised. Nor is the effect the AGN activity has on the growth of the central overdensities.

so basically we still dont have a clear understanding of the formation and high SFR of overdensities at HzRGs.

\section{Objectives}

The main Objective is to study about the overdensities of region 45arcmin2 around a high redshift radio galaxy sources (known as HzRGs) object, an example will be 3C266. overdensities means there are a lot of "objects" or galaxies around 3C266, and density is higher compared with other region in space. so i shall find the relation between the central object and the galaxies nearby. and some statistical study like surface number density.

\section{Introduction and background}

research has shown that in redshift of z = 2-3, there is a very high rate of star forming galaxies rate, the redshift zone is known as "cosmic noon". the reason is unknown. Why? (need to do some literature review). and the galaxies that show high SFR is called "dusty start forming galaxies (DSFGs)", dusty as due to the high concentration of gas contained in that galaxy. and due to dusty, it's not suitable to use radio and optical instrument to observe. however submm (microwave) is a good wavelength to observe (but why?) 

overdensities:
a region in space where the density of matter is relatively high than others. a lot of interesting things can happen at overdensities, rather than underdensities, where it will evolve into nothing, void. this proposal is to examine the overdensities region near the HzRG region.

star forming galaxies:
galaxies which is high in SFR, usually dusty cause of abundance in gas, not quite sure which gas though, neutral hydrogen?
then what's 450 and 850 micron gonna do? have to review on scuba 2 then!

Cosmic Noon:
That was when star formation in massive galaxies was at its peak

Scaling relation:
strong trends that are observed between important physical properties (such as mass, size, luminosity and colours) of galaxies. 
\url{https://astronomy.swin.edu.au/cosmos/S/Scaling+Relations#:~:text=Scaling%20relations%20describe%20strong%20trends%20that%20are%20observed,though%20different%20relations%20are%20used%20for%20each%20type.}

Spectral energy distribution (SED):
can derive photometric redshift, stellar mass and SFR

Spectral resolution (R) describes the ability of a sensor to define fine wavelength intervals

spectroscopy redshift is used to calibrate and optimize photometric redshift
Spitzer/MIPS 24 micron

\section{Research Methodology}
scuba 2 runs on 450 and 850micron, so around near infrared area, 
\section{Data Analysis and Expected Results}
\section{Significance}
\section{Conclusion}

\bibliographystyle{plain}
\bibliography{sample}

\end{document}